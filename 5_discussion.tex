
\section{Summary and significance of findings}

In this dissertation I have shown evidence for two hypotheses: that sensory selection may be a consistent computation, whether spatial or featural, and that it is a whole-brain phenomenon, involving not only change in sensory representation but changes in how information is gated from sensory areas to downstream regions. These findings force us to reconsider the common idea that attention is largely a matter of changing sensory representations to enhance perception, at the cost of unattended signals. We also must reconsider whether an arbitrary distinction should be made between different forms of selection -- the similar improvement in perceptual ability due to attention suggests that they may all be the result of a common computation. 

To demonstrate these findings I used linking models, computational models which make explicit the computations between sensory representation and perceptual decisions. Compared to simply measuring sensory representations during attentional behaviors, linking models are far more powerful: they force us to correctly match the scale of a change in representation to an improvement in behavior. This makes explicit the 'link function', or hypothesis about how attention is presumably improving our perceptual abilities. By using a linking model we showed that although sensory changes occur during attention to motion visibility, they don't account for all of the behavioral improvements which we observed. 