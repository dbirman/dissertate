To sample the important parts of the visual world observers make saccades, moving the high-resolution and color-sensitive fovea to informative locations. Choosing to make a saccade requires sampling the periphery and identifying potentially important parts of the visual scene. This \emph{covert attention}, without eye movement, is essential to selecting information in an efficient manner. At an intuitive level, covert attention is a focusing on a feature or a location in the visual world and, in tandem, a suppression of other irrelevant features and locations. When operationalized into the laboratory, cueing an observer with covert attention can be shown to result in improved detection, smaller thresholds of discrimination, faster reaction times, and suppression of distractors. These changes are known to be in part the result of small tweaks to the representation of visual stimuli in sensory cortex, but are also the result of context-dependent selection occurring after sensory processing has gone to completion. How attention implements this balance of sensory change and selection is a central problem for the neuroscience of vision. To investigate this balance, I measured how perception and cortical activity change during a behavior engaging selective visual attention. I showed using a computational linking model that changes in sensory cortex alone are insufficient to account for the improvements in behavior. Instead, both sensory change and flexible downstream readout are necessary components of selective visual attention.